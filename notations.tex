\documentclass[a4paper,12pt]{article}
\usepackage{mystyle}
\usepackage{commands}

\begin{document}
\section*{Notations}
\addcontentsline{toc}{section}{Notations}
We summarize some of the frequently used notation in this document.
Those already introduced in Mukres \cite{Munkres:Topology} are not included here.

Let \( P \) and \( Q \) be formulas,
let \( A \) and \( B \) be sets, and let \( f \) be a function.
\begin{enumerate}
	\renewcommand{\labelenumi}{\(\diamond\)}
	\item
	      \( \neg \)\COLON
	      negation.
	      
	\item
	      \( \vee \)\COLON
	      disjunction.
	      
	\item
	      \( \wedge \)\COLON
	      conjunction.
	      
	\item
	      \( \Rightarrow \)\COLON
	      implication.
	      
	\item
	      \( \Leftrightarrow \)\COLON
	      equivalence.
	      Formally,
	      \( P \Leftrightarrow Q \) is defined as
	      \( (P \Rightarrow Q) \wedge (Q \Rightarrow P)\).
	      
	\item
	      \( \equiv \)\COLON
	      the same meaning as "\( \Leftrightarrow \)."
	      
	\item
	      \( \Forall \)\COLON
	      universal quantifier.
	      
	\item
	      \( \Exists \)\COLON
	      existential quantifier.
	      
	\item
	      \( \unique \)\COLON
	      "\( \unique{ x }\left[ P(x) \right]\)"
	      means
	      "\( P(x) \) holds for exactly one \( x \)."
	      Formally,
	      \( \unique{ x }\left[ P(x) \right]\)
	      is defined as
	      \(\Exists{ x }
	      \left[ P(x) \wedge \Forall{ y }\left[ P(y) \Rightarrow x=y \right] \right]\).
	      
	      
	\item
	      \( T \)\COLON
	      short for "true."
	      
	\item
	      \( F \)\COLON
	      short for "false."
	      
	\item
	      \( := \)\COLON
	      "is defined to be."
	      \( a:=b \) means that \( a\) is defined to be \( b \), while 
	      \( a=:b \) means that \( b\) is defined to be \( a \).
	      
	\item
	      \( \setminus \)\COLON
	      the difference of two sets.
	      
	\item
	      \( \subsetneq \)\COLON
	      proper inclusion.
	      
	\item
	      \( \mapsto \)\COLON
	      Assignment.
	      We often write
	      "\( f:A \ni a \mapsto b \in B\)"
	      to mean \( f \) is a function from \( A \) to \( B \)
	      assigning, to each element \( a \) of \( A \), an element \( b \) of \( B \).
	      
	\item
	      \( \func{A}{B} \)\COLON
	      the set of all functions from \( A \) to \( B \).
	      
	\item
	      \( \inj{A}{B} \)\COLON
	      the set of all injective functions from \( A \) to \( B \).
	      
	\item
	      \( \surj{A}{B} \)\COLON
	      the set of all surjective functions from \( A \) to \( B \).
	      
	\item
	      \( \bij{A}{B} \)\COLON
	      the set of all bijective functions from \( A \) to \( B \).
	      
	\item
	      \( \sim \)\COLON
	      equivalence relation.
	      
	\item
	      \( \simeq \)\COLON
	      equivalence relation (often about order type in this document).
	      
	\item
	      \( \hookrightarrow \)\COLON
	      existence of injection.
	      We write \( A \hookrightarrow B\) if there exists an injection from \( A \) into \( B \).
	      
	\item
	      \( (A,<) \)\COLON
	      the set \( A \) equipped with the order relation \( < \).
	      
	\item
	      \( \min{A} \)\COLON
	      a smallest element of \( A \).
	      
	\item
	      \( \max{A} \)\COLON
	      a largest element of \( A \).
	      
	\item
	      \( \inf{A} \)\COLON
	      an infimum of \( A \).
	      
	\item
	      \( \sup{A} \)\COLON
	      a supremum of \( A \).
	      
	      
	      
\end{enumerate}

\end{document}