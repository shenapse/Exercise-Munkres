\documentclass[a4paper,12pt]{article}
\usepackage{mystyle}
\usepackage{commands}
\mathtoolsset{showonlyrefs=true}

\begin{document}
\section{The Maximum Principle}
\setcounter{exe}{0}
\begin{exe}
	If \( a \) and \( b \) are real numbers, define \( a \prec b \) if \( b-a \) is positive and rational.
	Show that this is a strict partial order on \( \mathbb{R} \).
	What are the maximal simply ordered subsets?
\end{exe}
\begin{sol}
	For any \( a \in \mathbb{R} \) we have \( a - a = 0 \),
	which gives \( \neg a \prec a \).
	Suppose \( a \prec b \) and \( b \prec c \), or equivalently
	suppose
	\( b -a \in \mathbb{Q}_{+} \)
	and
	\( c-b\in \mathbb{Q}_{+} \).
	Then we have
	\begin{equation*}
		c-a = (c-b) + (b-a) \in \mathbb{Q}_{+},
	\end{equation*}
	which leads to \( a \prec c \).
	Hence, \( \prec \) is a strict partial order on \( \mathbb{R} \).
	
	A maximal simply ordered subset is the set
	\begin{equation*}
		\left\{ b\in \mathbb{R} \pipe 
		\Exists{ q \in \mathbb{Q} }
		\left[ b= a+q \right]
		\right\},
	\end{equation*}
	where \( a \in \mathbb{R} \) is arbitrary given.
	\qed\end{sol}

\begin{exe}\leavevmode \par
	\begin{enumerate}
		\item
		      Let \( \prec \) be a strict partial order on the set \( A \).
		      Define a relation on \( A \) by letting \( a \preceq b \)
		      if either \( a \prec b \) or \( a=b \).
		      Show that this relation has the following properties, which are called the 
		      \textbf{\textit{partial order axioms}}:
		      \begin{enumerate}
			      \item
			            \( a \preceq a \) for all \( a \in A \).
			            
			      \item
			            \( a \preceq b \) and \( b \preceq a \Rightarrow a=b\). 
			      \item
			            \( a \preceq b \) and \( b \preceq c \Rightarrow a \preceq c\). 
		      \end{enumerate}
		\item
		      Let \( P \) be a relation on \( A \) that satisfies properties (i)-(iii).
		      Define a relation \( S \) on \( A \) by letting \( aSb \) if \( aPb \) and \( a \neq b \).
		      Show that \( S \) is a strict partial order on \( A \).
	\end{enumerate}
\end{exe}
\begin{sol}\leavevmode \par
	\fbox{(a)}
	(i) is obvious by the fact \( a=a \) for all \( a \in A \).
	To verify (ii), observe that, exploiting transitivity of \( \prec \), we have
	\begin{equation*}
		a\prec b \wedge b \prec a
		\equiv
		a \prec a \wedge a\prec b \wedge b \prec a
		\equiv
		F,
	\end{equation*}
	and that
	\begin{eqnarray*}
		a \preceq b \wedge b \prec a
		&\equiv&
		(a\prec b \vee a=b) \wedge (b\prec a \vee a=b)\\
		&\equiv&
		(a\prec b \wedge b\prec a) \vee a=b\\
		&\equiv&
		a=b.
	\end{eqnarray*}
	To see (iii),
	suppose \(  a\preceq b \) and \(  b\preceq c \).
	If both \( a \prec b \) and \( b \prec c \) hold, then transitivity of \( \prec \)
	gives \( a\prec c \).
	If not, it is easy to check \( a \preceq c \).
	
	\fbox{(b)}
	Nonreflexivity is proved by seeing
	\begin{eqnarray*}
		\Forall{ a\in A }\left[ aPa \right]
		&\equiv&
		\Forall{ a\in A }\left[ aPa \wedge a \neq a \right]\\
		&\equiv&
		\Forall{ a \in A }\left[ T \wedge F \right]\\
		&\equiv&
		F.
	\end{eqnarray*}
	To show transitivity,
	observe that (iii) on the partial order axioms gives
	\begin{equation*}
		aSb \wedge bSc \Rightarrow aPc.
	\end{equation*}
	Moreover, (ii) on the partial order axioms gives
	\begin{eqnarray*}
		aSb \wedge bSc \Rightarrow a=c
		&\equiv&
		aSb \wedge bSc \Rightarrow aSb \wedge bSa \wedge a=c\\
		&\equiv&
		aSb \wedge bSc \Rightarrow a=b \wedge a=c\\
		&\equiv&
		F,
	\end{eqnarray*}
	which implies
	\begin{equation*}
		aSb \wedge bSc \Rightarrow a\neq c.
	\end{equation*}
	Thus,
	\( aSb \wedge bSc \Rightarrow aSc \).
	\qed\end{sol}

\begin{exe}
	Let \( A \) be a strict partial order \( \prec \);
	let \(  x \in A \).
	Suppose that we wish to find a maximal simply ordered subset \( B \) of \( A \)
	that contains \( x \).
	One plausible way of attempting to define \( B \) is to let \( B \) equal the set of all those elements of \( A \) that are \textit{comparable} with \( x \);
	\begin{equation*}
		B=\left\{ y \in A \pipe x\prec y \vee y \prec x \right\}.
	\end{equation*}
	But this will not always work.
	In which of Example 1 and 2 will this procedure succeed and in which will it not?
\end{exe}
\begin{sol}
	It fails in Example 1.
	Consider
	\( A:=\left\{ \left\{ 1,2 \right\}, \left\{ 3 \right\},\mathcal{P}(\mathbb{Z}_{+}) \right\} \),
	and
	\( x:= \mathcal{P}(\mathbb{Z}_{+}) \),
	in which case we find
	\( B=\left\{ \left\{ 1,2 \right\}, \left\{ 3 \right\} \right\} \) not simply ordered.
	\qed\end{sol}

\begin{exe}
	Given two points \( (x_0,y_0) \) and \( (x_1,y_1) \) of \( \mathbb{R}^2 \),
	define
	\begin{equation*}
		(x_0,y_0) \prec (x_1,y_1) 
	\end{equation*}
	if \( x_0 < x_1 \) and \( y_0 \le y_1 \).
	Show that the curves \( y=x^3 \) and \( y=2 \) are maximal simply ordered subsets of \( \mathbb{R}^2 \),
	and the curve \( y=x^2 \) is not.
	Find all maximal simply ordered subsets.
\end{exe}
\begin{sol}
	We begin with considering the last question for convenience.
	We claim that
	\begin{enumerate}
		\item[(i)]
		      Let \( A \) be a subset of \( \mathbb{R}^2 \).
		      \( A \) is simply ordered by \( \prec \)
		      if and only if it is the rule of a weakly increasing function, denoted by \( f_A \), on a subset of \( \mathbb{R} \).
		      
		\item[(ii)]
		      Moreover,
		      \( (A,\prec) \) is a maximal simply ordered set if and only if the domain of \( f_A \) equals \( \mathbb{R} \).
	\end{enumerate}
	''if'' part of (i) is easy to check.
	Consider ''only if'' part.
	Suppose \( (x,y_0),(x,y_1) \in A \).
	Since these points are not comparable under \( \prec \), a simple order,
	comparability of it implies
	\( (x,y_0)=(x,y_1) \)
	and so,
	Hence, \( A \) is the rule of some function on a subset of \( \mathbb{R} \).
	It follows from the definition of \( \prec \) that
	so derived function is in fact weakly increasing.
	This completes the proof of (i).
	
	We proceed to ''if'' part of (ii).
	Suppose the domain of \( f_A \), denoted by \( D \),
	coincides with \( \mathbb{R} \).
	Then, for any \( (a,b)\in \mathbb{R}^2 \setminus A \),
	there exists \( (a,f_A(a))\in A \).
	Clearly, \( (a,b) \) and \( (a,f_A(a)) \) are not comparable,
	which implies \( A \cup \left\{ (a,b) \right\} \) is no longer a simply ordered set.
	So, \( A \) is maximal.
	
	We next show (contrapositive of) the converse.
	Suppose \( D \subsetneq \mathbb{R} \),
	and let \(  p \in \mathbb{R} \setminus D \).
	It is easy to construct a weakly increasing function \( g \) on \( D \cup \left\{ p \right\} \)
	that extends \( f_A \).
	Let \( G \) be the rule of \( g \).
	(i) tells us that \( G \) is simply ordered, and \( G \supsetneq A \),
	which means \( A \) is not a maximal simply ordered set.
	Thus, (ii) follows.
	
	We deduce from (i) and (ii) that the set of all maximal simply ordered subsets is the set of the rules of all weakly increasing functions from \( \mathbb{R} \) to \( \mathbb{R} \).
	
	It is now trivial to to verify the claim about \( y=x^3 \), \( y=2 \),
	and \( y=x^2 \).
	\qed\end{sol}

\begin{exe}
	Show that Zorn's Lemma implies the following:\\
	\text{Lemma (Kuratowski).}\;\;
	Let \( \mathcal{A} \) be a collection of sets.
	Suppose that for every subcollection \( \mathcal{B} \) of  \( \mathcal{A} \)
	that is simply ordered by proper inclusion, the union of the elements of
	\( \mathcal{B} \) belongs to \( \mathcal{A} \).
	Then \( \mathcal{B} \) has an element that is properly contained in no other element of \( \mathcal{A} \).
\end{exe}
\begin{sol}
	Proper inclusion induces a strict partial \( \prec \) order on \( \mathcal{A} \).
	Every simply ordered subcollection \( \mathcal{B} \) of \( \mathcal{A} \)
	admits an upper bound in \( \mathcal{A} \)
	since the union of the elements of \( \mathcal{B} \) belongs to \( \mathcal{A} \) by assumption,
	which means \( (\mathcal{A},\prec) \) satisfies the hypothesis of
	Zorn's Lemma.
	Hence, \( (\mathcal{A},\prec) \) has a maximal element,
	which is property contained in no other element of \( \mathcal{A} \)
	by definition.
	\qed\end{sol}

\begin{exe}
	A collection \( \mathcal{A} \) of subsets of a set \( X \) is said to be of \textit{finite type},
	provided that a subset \( B \) of \( X \) belongs to \( \mathcal{A} \) if and only if
	every finite subset of \( B \) belongs to \( \mathcal{A} \).
	Show that Kuratowski lemma implies the following:\\
	\text{Lemma (Tukey, 1940).}\;\;
	Let \( \mathcal{A} \) be a collection of sets.
	If \( \mathcal{A} \)  is of finite type,
	then \( \mathcal{A} \) has an element 
	that is properly contained in no other element of \( \mathcal{A} \).
\end{exe}
\begin{sol}
	Suppose \( \mathcal{A} \) be of finite type.
	Let \( \mathcal{B} \) be a subcollection of \( \mathcal{A} \) that is simply ordered by proper inclusion.
	Set \( B^{\ast}:=\bigcup_{B\in \mathcal{B}}B\).
	If \( F \) is a finite subset of \( B^{\ast} \),
	then there exists \( B\in \mathcal{B} \) such that \( F \subset B \).
	Since \( B\in \mathcal{A} \) and \( \mathcal{A} \) is assumed to be of finite type,
	we deduce \( F \in \mathcal{A} \),
	which leads to \( B^{\ast} \in \mathcal{A} \).
	Thus, Kuratowski lemma implies Tukey lemma.
	\qed\end{sol}

\begin{exe}
	Show that Tukey lemma implies the Hausdorff maximum principle.
\end{exe}
\begin{sol}
	Let \( (A,\prec) \) be a set \( A \) equipped with a strict partial order \( \prec \),
	and let \( \mathcal{A} \) be the collection of all subsets of \( A \)
	that are simply ordered by \( \prec \).
	It suffices to show that \( \mathcal{A} \) is of finite type  thanks to Tukey lemma.
	
	\S3 Exercise 7 implies that if \( B\in \mathcal{A} \), then every (finite) subset of \( B \) belongs to \( \mathcal{A}\).
	
	Conversely, let \( B \) be a subset of \( A \) of which every finite subset belongs to \( \mathcal{A} \).
	We insist \( B \in \mathcal{A} \).
	For, if \( x \) and \( y \) are two distinct points in \( B \),
	then the fact \( \left( \left\{ x,y \right\},\prec_{\left\{ x,y \right\}} \right)\)
	is a simply ordered set gives that
	\( B \) satisfies comparability,
	where \( \prec_{\left\{ x,y \right\}} \) is a restriction of \( \prec \) onto
	\( \left\{ x,y \right\}\).
	Similarly, we deduce that \( B \) satisfies nonareflexivity and transitivity,
	and so \( B \) is simply ordered.
	This means \( B \in \mathcal{A} \).
	Thus, \( \mathcal{A} \) is of finite type.
	\qed\end{sol}

\begin{exe}
	A typical use of Zorn's lemma in algebra is the proof that every vector space has a basis.
	Recall that if \( A \) is a subset of the vector space \( V \),
	we say a vector belongs to the \textit{span} of \( A \)
	if it equals a finite linear combination of elements of \( A \).
	The set \( A \) is \textit{independent} if the only finite linear combination of elements of \( A \) that equals zero vector is the trivial one having all coefficient zero.
	If \( A \) is independent and if every vector in \( V \) belongs to the span of \( A \),
	then \( A \) is a \textit{basis} of \( V \).
	\begin{enumerate}
		\item
		      If \( A \) is independent and \( v \in V \) does not belong to the span of \( A \),
		      show \( A \cup \left\{ v \right\} \) is independent.
		      
		\item
		      Show the collection \( \mathcal{I} \) of all independent sets in \( V \) has a maximum element.
		      
		\item
		      Show that \( V \) has a basis.
	\end{enumerate}
\end{exe}
\begin{sol}\leavevmode \par
	\fbox{(a)}
	Let \( A \) is independent  and let \( v \in V \).
	Suppose \( A \cup \left\{ v \right\} \) is not independent.
	There exist nontrivial coefficients \( c_1,c_2,\cdots,c_n \)
	such that
	\begin{equation}\label{eq:dependent_linear}
		c_1 a_1 + \cdots + c_{n-1}a_{n-1} + c_n v = 0.
	\end{equation}
	We may assume all \( c_n \) are nonzero.
	It then follows from (\refeq{eq:dependent_linear}) that
	\( v \) belongs to the span of \( A \),
	from which we deduce that
	\( v \in V \) belongs to the span of \( A \).
	
	\fbox{(b)}
	Introduce a strict partial order \( \prec \) on \( \mathcal{I} \) by setting
	\( A \prec B \) if \( A \subsetneq B \).
	Then, Hausdorff maximum principle yields a required maximal element.
	
	\fbox{(c)}
	Let \( I \in \mathcal{I} \), and let \( v \) be a vector in \( V \).
	Suppose \( v \) does not belong to the span of \( I \).
	Then (a) tells us that \( I \cup \left\{ v \right\} \) is independent.
	This means \( I \) is not a maximal element of \( \mathcal{I} \).
	Thus, every vector in \( V \) belongs to the span of some maximal element of \( \mathcal{I} \).
	\qed\end{sol}

\end{document}